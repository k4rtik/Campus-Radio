\documentclass[12pt,a4paper]{article}
\usepackage{url} %for proper url entries
\usepackage[bookmarks, colorlinks=false, pdfborder={0 0 0}, pdftitle={Campus Radio Portal - Description}, pdfauthor={Aviral and Kartik}, pdfsubject={DBMS Project}, pdfkeywords={dbms, project, radio}]{hyperref} %for creating links in the pdf version and other additional pdf attributes, no effect on the printed document
\usepackage{fullpage}
\usepackage{nopageno}

\begin{document}
\title{Campus Radio Portal}
\author{Kartik Singhal - B090566CS\\
	Aviral Nigam - B090871CS}
\date{\today}  %\today is replaced with the current date
\maketitle


\paragraph*{Role of each group member:}
Both of us will work on the design, implementation and testing of the proposed project.
\paragraph*{About the project:}
In this project we are designing a campus radio portal which will help broadcast radio shows on intranet (LAN) via live streaming. The portal will enable the visitors to browse through the archive of the shows, comment on them, read about the RJs, participate in polls and discussion, etc. This community portal will also serve as a place for the campus radio team to engage in discussion about future shows, receive feedback on their work, make news \& announcements, etc.

\paragraph*{Source of the project idea:}
The idea of campus radio portal occured to us from 3 final year students - Girish Krishnan, Dinesh and Aravindhan - who are working to setup an intranet-based radio system in NIT Calicut campus.

\paragraph*{Platform:}
We will be using Linux, Apache, PostgreSQL and PHP for the backend. For the front end, a usual web browser on any platform will suffice for most of the functionality of the portal. An exception will be some plugin requirement for streaming the live audio during the show timing.

\paragraph*{Current status of the project:}
The feature requirements from the campus radio team has been collected and discussion on the design of the database has been started. Work going on for doing a test setup of live streaming over the intranet.

\begin{thebibliography}{9}
	\bibitem{las}
	  Linux Action Show,\ \url{http://jupiterbroadcasting.com}
	\bibitem{ruia}
	  Ruia College Radio,\ \url{http://ruiacollegeradio.com}
	\bibitem{maska}
	  Radio Maska,\ \url{http://radiomaska.com}
\end{thebibliography}

\end{document}